%%%%%%%%%%%%%%%%%%%%%%%%%%%%%%%%%%%%%%%%%
% Medium Length Professional CV
% LaTeX Template
% Version 2.0 (8/5/13)
%
% This template requires the resume.cls file to be in the same directory as the
% .tex file. The resume.cls file provides the resume style used for structuring the
% document.
%
%%%%%%%%%%%%%%%%%%%%%%%%%%%%%%%%%%%%%%%%%

%----------------------------------------------------------------------------------------
%	PACKAGES AND OTHER DOCUMENT CONFIGURATIONS
%----------------------------------------------------------------------------------------

\documentclass{resume} % Use the custom resume.cls style
\usepackage{hyperref}

\usepackage[left=0.75in,top=0.6in,right=0.75in,bottom=0.6in]{geometry} % Document margins
\newcommand{\tab}[1]{\hspace{.2667\textwidth}\rlap{#1}}
\newcommand{\itab}[1]{\hspace{0em}\rlap{#1}}
\name{Yitian Wang} % Your name
%\address{\href{http://linkedin.com/in/tim-wang-yitian/}{http://linkedin.com/in/tim-wang-yitian/}} % Your address
%\address{123 Pleasant Lane \\ City, State 12345} % Your secondary addess (optional)
%\address{ywang1057@ucr.edu \\ 929-624-1021 \\ http://linkedin.com/in/tim-wang-yitian} % Your phone number and email
\address{ywang1057@ucr.edu \\ http://cosmotim.github.io \\ http://linkedin.com/in/tim-wang-yitian} % Your phone number and email
%\address{http://cosmotim.github.io} % Your phone number and email
\begin{document}

%----------------------------------------------------------------------------------------
%	EDUCATION SECTION
%----------------------------------------------------------------------------------------

\begin{rSection}{EDUCATION}
{\bf University of California, Riverside, CA}\hfill{\em Mar. 2021 - Dec. 2025 (expected)}
\\ Ph.D. in Electrical and Computer Engineering
\\ Award: MRS 2023 Spring Highly Commended Student Talk Award
\\
\\{\bf Columbia University in the City of New York, NY}\hfill{\em Aug. 2019 - Feb. 2021} 
\\ Master of Science, Material science and engineering
\\
\\{\bf University of California, Berkeley, CA}\hfill{\em Jul. 2017 - Aug. 2017}
\\ Summer Session
\\
\\{\bf Beijing Normal University, CN} \hfill {\em Sep. 2015 - Jun. 2019} 
\\ Bachelor of Science, Physics
\\ Award: 'Jingshi' Scholarship (2016 - 2018)

\end{rSection}


\begin{rSection}{Skills }

	\textbf{Software / Programming}
\begin{itemize}
    \item Python (advanced), MATLAB (advanced), LaTeX (advanced), SolidWorks (intermediate)
    \item GSAS-II (advanced), Mantid (advanced), ImageJ (intermediate)
\end{itemize}

	\textbf{Experimental Techniques}
\begin{itemize}
    \item FZ Crystal Growth (advanced), PVD (intermediate), CVD (intermediate)
    \item SEM (advanced), XRD (advanced), INS (advanced), Raman (intermediate), FTIR (intermediate)
    \item PPMS (advanced), DSC (intermediate), TGA (intermediate)
    \item Battery Assembly (advanced), Impedance Spectroscopy (intermediate)
\end{itemize}

\end{rSection}

%--------------------------------------------------------------------------------
%    Projects And Seminars
%-----------------------------------------------------------------------------------------------
% \begin{rSection}{RESEARCH EXPERIENCE}

% \begin{rSubsection}
% {Thermal transport in solid-state battery}{\em Mar. 2021 - Now}
% {University of California}{Riverside, CA}
% \item Studied the thermal transport in various types of solid-electrolyte and electrode materials.
% \end{rSubsection}

% % \begin{rSubsection}
% % {Low temperature uniaxial strain device}{\em Sep. 2018 - May 2019}
% % {Beijing Normal University}{Beijing, CN}
% % \item Modeled three methods in Solidworks for applying uniaxial pressure on thin-film sampled with thermal expansion.
% % \item Tested the strain induced detwinning effect and processed the data with MatLab.
% % \end{rSubsection}

% % \begin{rSubsection}
% % {Atomic simulation in a nanotube system}{\em Jul. 2017 - Jun. 2018}
% % {Beijing Normal University}{Beijing, CN}
% % \item Calculated the energy gap between iron-carbon layers using DFT method.
% % \item Simulated the motion of atoms with self made cellular automaton model in MatLab.
% % \end{rSubsection}
% \end{rSection}

\begin{rSection}{PROJECTS}
    
\begin{rSubsection}
    {Thermal transport and ionic mobility in solid electrolytes}{\em Jun. 2021 - Jun. 2025}
    {University of California}{Riverside, CA}
\item Grew single crystal samples with floating zone method using image furnace.
\item Studied the physical properties in comparison with other solid electrolytes.
\end{rSubsection}

\begin{rSubsection}
    {Inelastic neutron scattering on LLZTO single crystal}{\em Jan. 2022 - Jul. 2023}
    {Oak Ridge National Laboratory}{Oak Ridge, TN}
    \item Conducted inelastic neutron scattering experiments using TAX and ARCS.
    \item Processed and analyzed the data with Mantid and Python.
\end{rSubsection}

\begin{rSubsection}
    {Two-channel fitting model for thermal conductivity}{\em Mar. 2022 - Sep. 2023}
    {University of California}{Riverside, CA}
\item Developed a expandable fitting model for thermal conductivity data in MatLab.
\end{rSubsection}

\begin{rSubsection}
{Advanced separator of lithium-ion battery}{\em Oct. 2019 - Jan. 2021}
{Columbia University}{New York, NY}
 \item Made composite battery separator with $\gamma-C_3 N_4$ and PVDF.
 \item Assembled full cell batteries and tested electrochemical properties.
\end{rSubsection}

% \begin{rSubsection}
% {Impedence calculator for battery system with user interface}{\em Nov.2019}
% {Columbia university}{New York, NY}
% \item Set up a user interface with the GUI package in MatLab.
% % \item Calculated the impedance in complex representation.
% \end{rSubsection}

\begin{rSubsection}
{First-principle calculation of olivine structure }{\em Sep. 2019 - Jan. 2020}
{Columbia university}{New York, NY}
\item Calculated the stable olivine structure with hydrogen defects with Quantum ESPRESSO.
\end{rSubsection}


\begin{rSubsection}
{Low temperature uniaxial strain device}{\em Sep. 2018 - May 2019}
{Beijing Normal University}{Beijing, CN}
\item Modeled devices in Solidworks to apply uniaxial pressure on film samples utilizing thermal expansion.
\item Tested the strain induced detwinning effect with PPMS.
\end{rSubsection}

\end{rSection}


%----------------------------------------------------------------------------------------
%	TECHNICAL STRENGTHS SECTION
%----------------------------------------------------------------------------------------




%----------------------------------------------------------------------------------------
%	WORK EXPERIENCE SECTION
%----------------------------------------------------------------------------------------




%	EXAMPLE SECTION
%----------------------------------------------------------------------------------------


%	Publications
%----------------------------------------------------------------------------------------
\begin{rSection}{SELECTED PUBLICATIONS}
% Publications formatted with bold title and details as subsection
\begin{itemize}
\item \href{https://doi.org/10.1039/D5CC04693A}{Glass-like thermal transport in polycrystalline perovskite lithium-ion conductor \\ Li$_{3/8}$Sr$_{7/16}$Hf$_{1/4}$Ta$_{3/4}$O$_3$}
    \begin{itemize}
        \item Chemical Communications, 2025
        \item \textbf{Y. Wang}, Q. Jia, S. Li, L. Shi, Y. Li, X. Chen
    \end{itemize}

\item \href{https://doi.org/10.1007/s42864-025-00357-6}{Low Thermal Conductivity and Lattice Anharmonicity of NaSICON-type Solid Electrolyte \\ Na$_3$Zr$_2$Si$_2$PO$_{12}$}
    \begin{itemize}
        \item Tungsten, 2025
        \item \textbf{Y. Wang}, Q. Jia, S. Li, L. Shi, Y. Li, X. Chen
    \end{itemize}

\item \href{https://doi.org/10.1103/6wj2-kzhh}{Origin of intrinsically low thermal conductivity in a garnet-type solid electrolyte: Linking lattice and ionic dynamics with thermal transport}
    \begin{itemize}
        \item PRX Energy, 2025
        \item \textbf{Y. Wang}, Y. Su, J. Carrete, H. Zhang, N. Wu, Y. Li, H. Li, J. He, Y. Xu, S. Guo, Q. Cai, D. L. Abernathy, T. Williams, K. V. Kravchyk, M. V. Kovalenko, G. K. H. Madsen, C. Li, X. Chen
    \end{itemize}

\item \href{https://doi.org/10.1039/D4TA02264E}{Thermal properties and lattice anharmonicity of Li-ion conducting garnet solid electrolyte \\ Li$_{6.5}$La$_3$Zr$_{1.5}$Ta$_{0.5}$O$_{12}$}
    \begin{itemize}
        \item Journal of Materials Chemistry A, 2024
        \item \textbf{Y. Wang}, S. Li, N. Wu, Q. Jia, T. Hoke, L. Shi, Y. Li, X. Chen
    \end{itemize}

\item \href{https://doi.org/10.1063/5.0214897}{Enhanced magnon thermal transport in yttrium-doped spin ladder compounds Sr$_{14-x}$Y$_x$Cu$_{24}$O$_{41}$}
    \begin{itemize}
        \item Journal of Applied Physics, 2024
        \item S. Li, S. Guo, \textbf{Y. Wang}, H. Li, Y. Xu, V. Carta, J. Zhou, X. Chen
    \end{itemize}

\item \href{https://doi.org/10.1016/j.xcrp.2024.101879}{Size-dependent magnon thermal transport in a nanostructured quantum magnet}
    \begin{itemize}
        \item Cell Reports Physical Science, 2024
        \item S. Guo, H. Li, X. Bai, \textbf{Y. Wang}, S. Li, R. E. Dunin-Borkowski, J. Zhou, X. Chen
    \end{itemize}

\item \href{https://doi.org/10.1021/acs.chemmater.2c02155}{Crystal structure and thermoelectric properties of layered van der Waals semimetal ZrTiSe$_4$}
    \begin{itemize}
        \item Chemistry of Materials, 2022
        \item Y. Xu, Z. Barani, P. Xiao, S. Sudhindra, \textbf{Y. Wang}, A. A. Rezaie, V. Carta, K. N. Bozhilov, D. Luong, B. P. T. Fokwa, F. Kargar, A. A. Balandin, X. Chen
    \end{itemize}

\item \href{https://doi.org/10.1007/s42864-022-00176-z}{Single crystal growth and electrochemical studies of garnet-type fast Li-ion conductors}
    \begin{itemize}
        \item Tungsten, 2022
        \item \textbf{Y. Wang}, X. Chen
    \end{itemize}
\end{itemize}

\end{rSection}


% ----------------------------------------------------------------------------------------
% Extra Curricular
% ----------------------------------------------------------------------------------------

\begin{rSection}{EXTRA-CIRRUCULAR}
    
    \begin{qSubsection}
        {GSA Python Coding Contest}{Aug. 2020}
    \end{qSubsection}

    \begin{qSubsection}
        {COMAP Mathematical Contest in Modeling}{Jan. 2017}
    \end{qSubsection}

    \begin{qSubsection}
        {BNU Physics Department Soccer Team}{2015 - 2019}
    \end{qSubsection}
\end{rSection}

 \end{document}
