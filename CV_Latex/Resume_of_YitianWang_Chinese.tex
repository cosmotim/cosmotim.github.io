%%%%%%%%%%%%%%%%%%%%%%%%%%%%%%%%%%%%%%%%%
% Medium Length Professional CV (Chinese Version)
% LaTeX Template
% Version 2.0 (Chinese Translation)
%
% This template requires the resume.cls file to be in the same directory as the
% .tex file. The resume.cls file provides the resume style used for structuring the
% document.
%
%%%%%%%%%%%%%%%%%%%%%%%%%%%%%%%%%%%%%%%%%

%----------------------------------------------------------------------------------------
%	PACKAGES AND OTHER DOCUMENT CONFIGURATIONS
%----------------------------------------------------------------------------------------

\documentclass{resume} % Use the custom resume.cls style
\usepackage{xeCJK} % Support for Chinese characters
\usepackage{hyperref}


% Set Chinese fonts (you may need to adjust these based on your system)
\setCJKmainfont{STSong} % or use {STSong} on Mac, {NSimSun} on Windows
\setCJKsansfont{STHeiti} % or use {} on Mac

\usepackage[left=0.75in,top=0.6in,right=0.75in,bottom=0.6in]{geometry} % Document margins
\newcommand{\tab}[1]{\hspace{.2667\textwidth}\rlap{#1}}
\newcommand{\itab}[1]{\hspace{0em}\rlap{#1}}
\name{王倚天} % Your name in Chinese
\address{ywang1057@ucr.edu \\ \href{http://cosmotim.github.io}{http://cosmotim.github.io} \\ \href{http://linkedin.com/in/tim-wang-yitian/}{http://linkedin.com/in/tim-wang-yitian/}}
\begin{document}


%----------------------------------------------------------------------------------------
%	EDUCATION SECTION
%----------------------------------------------------------------------------------------

\begin{rSection}{教育背景}
{\bf 加州大学河滨分校}\hfill{\em 2021年3月 - 2026年3月(预期)}
\\ 电气与计算机工程博士学位
\\ MRS 2023春季会议 优秀学生报告奖
\\
\\{\bf 哥伦比亚大学,纽约}\hfill{\em 2019年8月 - 2021年2月} 
\\ 材料科学与工程理学硕士
\\
\\{\bf 加州大学伯克利分校}\hfill{\em 2017年7月 - 2017年8月}
\\ 暑期学校
\\
\\{\bf 北京师范大学,中国} \hfill {\em 2015年9月 - 2019年6月} 
\\ 物理学理学学士
\\ 京师奖学金(2016 - 2018)

\end{rSection}

\begin{rSection}{技能专长}

	\textbf{软件/编程}
\begin{itemize}
    \item Python(精通),MATLAB(精通)
    \item LaTeX(精通),SolidWorks(中等),VASP(中等),Quantum ESPRESSO(中等)
    \item GSAS-II(精通),Mantid(中等),ImageJ(中等)
\end{itemize}

	\textbf{实验技术}
\begin{itemize}
    \item 浮区法晶体生长(精通),物理气相沉积(中等),化学气相沉积(中等)
    \item 扫描电子显微镜(精通),X射线衍射(精通),非弹性中子散射(精通),拉曼光谱(中等),红外光谱(中等)
    \item 综合物性测量系统(PPMS)(精通),差示扫描量热法(精通),热重分析(中等)
    \item 电池组装(精通),电池阻抗谱(中等)
\end{itemize}

\end{rSection}

\begin{rSection}{项目经历}
    
\begin{rSubsection}
    {固态电解质中的热传输和离子迁移率研究}{\em 2021年6月 - 2025年6月}
    {加州大学河滨分校}{加利福尼亚州,美国}
\item 使用图像炉通过浮区法生长单晶样品。
\item 研究物理性质并与其他固态电解质进行比较。
\end{rSubsection}

\begin{rSubsection}
    {LLZTO单晶的非弹性中子散射研究}{\em 2022年1月 - 2023年7月}
    {橡树岭国家实验室}{田纳西州,美国}
    \item 使用TAX和ARCS进行非弹性中子散射实验。
    \item 使用Mantid和Python处理和分析数据。
\end{rSubsection}

\begin{rSubsection}
    {热导率双通道拟合模型}{\em 2022年3月 - 2023年9月}
    {加州大学河滨分校}{加利福尼亚州,美国}
\item 在MatLab中开发了热导率数据的可扩展拟合模型。
\end{rSubsection}

\begin{rSubsection}
{锂离子电池先进隔膜}{\em 2019年10月 - 2021年1月}
{哥伦比亚大学}{纽约,美国}
 \item 制备了$\gamma-C_3 N_4$和PVDF复合电池隔膜。
 \item 组装全电池并测试电化学性能。
\end{rSubsection}

\begin{rSubsection}
{橄榄石结构的第一性原理计算}{\em 2019年9月 - 2020年1月}
{哥伦比亚大学}{纽约,美国}
\item 使用Quantum ESPRESSO计算含氢缺陷的稳定橄榄石结构。
\end{rSubsection}

\begin{rSubsection}
{低温单轴应变装置}{\em 2018年9月 - 2019年5月}
{北京师范大学}{中国,北京}
\item 在Solidworks中建模装置,利用热膨胀对薄膜样品施加单轴应力。
\item 使用PPMS测试应变诱导的去孪晶效应。
\end{rSubsection}

\end{rSection}

%----------------------------------------------------------------------------------------
%	Publications
%----------------------------------------------------------------------------------------
\begin{rSection}{主要发表论文}

\begin{itemize}
\item \href{https://doi.org/10.1039/D5CC04693A}{Glass-like thermal transport in polycrystalline perovskite lithium-ion conductor \\ Li$_{3/8}$Sr$_{7/16}$Hf$_{1/4}$Ta$_{3/4}$O$_3$}
    \begin{itemize}
        \item Chemical Communications, 2025
        \item \textbf{Y. Wang}, Q. Jia, S. Li, L. Shi, Y. Li, X. Chen
    \end{itemize}

\item \href{https://doi.org/10.1007/s42864-025-00357-6}{Low Thermal Conductivity and Lattice Anharmonicity of NaSICON-type Solid Electrolyte \\ Na$_3$Zr$_2$Si$_2$PO$_{12}$}
    \begin{itemize}
        \item Tungsten, 2025 (Accepted)
        \item \textbf{Y. Wang}, Q. Jia, S. Li, L. Shi, Y. Li, X. Chen
    \end{itemize}

\item \href{https://doi.org/10.1103/6wj2-kzhh}{Origin of intrinsically low thermal conductivity in a garnet-type solid electrolyte: Linking lattice and ionic dynamics with thermal transport}
    \begin{itemize}
        \item PRX Energy, 2025
        \item \textbf{Y. Wang}, Y. Su, J. Carrete, H. Zhang, N. Wu, Y. Li, H. Li, J. He, Y. Xu, S. Guo, Q. Cai, D. L. Abernathy, T. Williams, K. V. Kravchyk, M. V. Kovalenko, G. K. H. Madsen, C. Li, X. Chen
    \end{itemize}

\item \href{https://doi.org/10.1039/D4TA02264E}{Thermal properties and lattice anharmonicity of Li-ion conducting garnet solid electrolyte \\ Li$_{6.5}$La$_3$Zr$_{1.5}$Ta$_{0.5}$O$_{12}$}
    \begin{itemize}
        \item Journal of Materials Chemistry A, 2024
        \item \textbf{Y. Wang}, S. Li, N. Wu, Q. Jia, T. Hoke, L. Shi, Y. Li, X. Chen
    \end{itemize}

\item \href{https://doi.org/10.1063/5.0214897}{Enhanced magnon thermal transport in yttrium-doped spin ladder compounds Sr$_{14-x}$Y$_x$Cu$_{24}$O$_{41}$}
    \begin{itemize}
        \item Journal of Applied Physics, 2024
        \item S. Li, S. Guo, \textbf{Y. Wang}, H. Li, Y. Xu, V. Carta, J. Zhou, X. Chen
    \end{itemize}

\item \href{https://doi.org/10.1016/j.xcrp.2024.101879}{Size-dependent magnon thermal transport in a nanostructured quantum magnet}
    \begin{itemize}
        \item Cell Reports Physical Science, 2024
        \item S. Guo, H. Li, X. Bai, \textbf{Y. Wang}, S. Li, R. E. Dunin-Borkowski, J. Zhou, X. Chen
    \end{itemize}

\item \href{https://doi.org/10.1021/acs.chemmater.2c02155}{Crystal structure and thermoelectric properties of layered van der Waals semimetal ZrTiSe$_4$}
    \begin{itemize}
        \item Chemistry of Materials, 2022
        \item Y. Xu, Z. Barani, P. Xiao, S. Sudhindra, \textbf{Y. Wang}, A. A. Rezaie, V. Carta, K. N. Bozhilov, D. Luong, B. P. T. Fokwa, F. Kargar, A. A. Balandin, X. Chen
    \end{itemize}

\item \href{https://doi.org/10.1007/s42864-022-00176-z}{Single crystal growth and electrochemical studies of garnet-type fast Li-ion conductors}
    \begin{itemize}
        \item Tungsten, 2022
        \item \textbf{Y. Wang}, X. Chen
    \end{itemize}
\end{itemize}


\end{rSection}

% ----------------------------------------------------------------------------------------
% Extra Curricular
% ----------------------------------------------------------------------------------------

\begin{rSection}{课外活动}
    
    \begin{qSubsection}
        {研究生会Python编程竞赛}{2020年8月}
    \end{qSubsection}

    \begin{qSubsection}
        {美国大学生数学建模竞赛}{2017年1月}
    \end{qSubsection}

    \begin{qSubsection}
        {北师大物理系足球队}{2015年 - 2019年}
    \end{qSubsection}
\end{rSection}

 \end{document}